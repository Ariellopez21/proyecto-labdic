\section{Título}

Desarrollo e implementación de sistema de gestión de préstamos e inventario del laboratorio del departamento de ingeniería en computación. 

\section{Antecedentes}

Actualmente, los préstamos en el laboratorio de computación se gestionan de forma manual con registros en papel, lo que exige revisiones diarias y comunicación individual con los usuarios para controlar devoluciones o extensiones. Paralelamente, el inventario se administra en una planilla Excel aislada, sin integración con los préstamos ni actualización automática, lo que genera errores y falta de precisión.
Ante esta situación, se requiere un sistema unificado que integre control de inventario y gestión de préstamos, con funciones de recordatorios, seguimiento y reportes que optimicen la organización y reduzcan errores en la administración del laboratorio.

\section{Objetivos}

\subsection{Objetivo General}
Desarrollar un sistema de gestión para el laboratorio de computación que administre de forma eficiente los préstamos de material y el control de inventario, optimizando los procesos y reduciendo los errores de la gestión manual.

\subsection{Objetivos Específicos}
\begin{itemize}
    \item Implementar un módulo de registro de préstamos en tiempo real, incluyendo usuario, material solicitado y fechas de entrega y devolución.
    \item Diseñar un inventario dinámico que se actualice automáticamente con cada préstamo o devolución, asegurando consistencia con el stock real.
    \item Incorporar un sistema de notificaciones y recordatorios para alertar a usuarios y administradores sobre vencimientos y extensiones de préstamos.
    \item Facilitar la búsqueda y generación de reportes sobre préstamos activos, historial de usuarios y disponibilidad de materiales.
    \item Proveer una interfaz de usuario intuitiva que simplifique el acceso a las principales funcionalidades del sistema.
    \item Garantizar la seguridad y el control de acceso mediante autenticación confiable y permisos diferenciados por roles (Estudiante, Profesor, Administrador, Invitado).
    \item Asegurar la mantenibilidad y escalabilidad del sistema mediante documentación adecuada y diseño modular.
\end{itemize}


\section{Avances}

El sistema cuenta con un estado inicial sólido en su estructura fundamental. El modelo de base de datos está mayormente implementado con soporte para control de versiones, y ya se dispone de módulos básicos de \textbf{Usuarios} y \textbf{Autenticación}, incluyendo cifrado seguro de contraseñas.  
Además, la arquitectura inicial separa \textbf{backend} y \textbf{frontend}, conectados mediante APIs funcionales, aunque con un avance global menor al 10\%.

\subsection*{Tecnologías utilizadas}
\begin{itemize}
    \item \textbf{Backend:} Python, SQLAlchemy, Alembic, pwdlib (hashing de contraseñas), APIs REST.
    \item \textbf{Frontend:} PrimeVue (interfaz de usuario), TypeScript (lógica y modularidad).
\end{itemize}

\section{Actividades}

\subsection*{Plan de Trabajo}

El desarrollo del sistema se organizará en hitos sucesivos que reflejan avances significativos en la aplicación.  
Se considera una dedicación de 3 horas semanales entre el 18 de agosto y el 30 de noviembre de 2025, exceptuando la semana del 18 de septiembre. Esto corresponde a 14 semanas efectivas, con un total aproximado de 42 horas de trabajo.

\subsection*{Etapas y Plazos}

\begin{itemize}
    \item \textbf{Hito 1 (18–24 agosto, 1 semana):} Análisis de requerimientos y planificación general.
    \item \textbf{Hito 2 (25 agosto–7 septiembre, 2 semanas):} Diseño y validación del modelo de base de datos.
    \item \textbf{--- Semana del 15–21 septiembre:} Pausa programada sin avances.
    \item \textbf{Hito 3 (8–21 septiembre, 2 semanas efectivas):} Implementación del módulo de usuarios (registro, autenticación, roles).
    \item \textbf{Hito 4 (22 septiembre–12 octubre, 3 semanas):} Desarrollo del módulo de inventario con actualización dinámica.
    \item \textbf{Hito 5 (13–26 octubre, 2 semanas):} Implementación del módulo de préstamos e integración con inventario.
    \item \textbf{Hito 6 (27 octubre–9 noviembre, 2 semanas):} Incorporación de notificaciones y herramientas de administración.
    \item \textbf{Hito 7 (10–30 noviembre, 2 semanas):} Marcha blanca, despliegue y entrega final del sistema.
\end{itemize}

\section{Necesidades}
Equipo personal del estudiante, conexión a internet, permiso de alojamiento de la aplicación en los servidores del departamento.

\section{Bitácora de control}
A través de informe realizado durante el transcurso del desarrollo de la aplicación.

%\section{Bibliografía}
%Básica del trabajo.

%\section{Patrocinante}
%Si corresponde.
