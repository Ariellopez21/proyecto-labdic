\section{Introducción}

El laboratorio del Departamento de Ingeniería en Computación cuenta con diversos procesos de gestión administrativa, entre los cuales destaca el préstamo de material como notebooks, periféricos, libros de estudio, adaptadores, entre otros más. Actualmente, esta tarea se realiza mediante formularios impresos y registros manuales, lo que dificulta el control, genera duplicidad de información y limita la trazabilidad de los préstamos.

Además, el laboratorio dispone de un documento Excel con el registro de tecnología inventariable, pero este permanece inactivo y no se utiliza en la práctica, lo que evidencia la falta de integración entre los registros físicos y digitales. 

Ante esta situación, surge la necesidad de implementar un sistema interno que digitalice los procesos de préstamo, notifique a encargados y solicitantes sobre las solicitudes activas, y mantenga actualizada la base de datos del inventario. La propuesta consiste en desarrollar una aplicación web basada en tecnologías de bases de datos relacionales, con arquitectura modular y escalable, que permita optimizar la gestión del laboratorio y sentar las bases para futuras mejoras.
