\section{Conclusiones}

La experiencia acumulada al trabajar de manera sostenida en un mismo proceso permite identificar con claridad los puntos críticos que dificultan su eficiencia. Cuando alguien trabaja durante un tiempo prolongado en una misma tarea, es capaz de identificar dónde se encuentran los puntos críticos. En este sentido, un ingeniero debe tener la capacidad de reconocer dichas oportunidades de mejora, lo que constituye la motivación principal que dio origen a esta propuesta.   

La formación recibida durante la carrera ha entregado la mayoría de las herramientas tecnológicas que se emplearán en la construcción del sistema. A esto se suma la investigación realizada para validar los motivos de selección de las tecnologías, sustentada en el marco teórico. De esta combinación, formación académica y fundamentación investigativa, nace la decisión de adoptar las tecnologías finalmente consideradas en el proyecto.  

Asimismo, los requerimientos del sistema constituyen un aspecto esencial para delimitar qué funcionalidades debe y no debe tener la aplicación. Estos no solo definen el alcance del desarrollo, sino que también permiten visualizar de manera ordenada dónde se ubican las distintas capacidades que el sistema ofrece, aun cuando algunas no sean perceptibles a simple vista o no tengan una representación visual directa.  

Finalmente, la planificación se organiza en un formato de hitos, que estructuran el trabajo y funcionan como puntos de control dentro del proceso. Este enfoque asegura que cada etapa cuente con un seguimiento adecuado y que el resultado final refleje fielmente lo planteado desde el inicio.
