\section{Conclusión}

El modelo de base de datos desarrollado cumple con los principios fundamentales de modularidad, coherencia estructural y buenas prácticas de diseño. Las tablas fueron definidas de manera clara y consistente, reflejando adecuadamente los procesos del sistema de inventario y permitiendo una gestión precisa tanto de usuarios como de productos y solicitudes de préstamo. El diagrama Entidad–Relación confirma la solidez del esquema y la correcta articulación entre sus componentes.

Asimismo, la integración de SQLAlchemy y Alembic provee un marco robusto para la evolución continua del sistema, asegurando que cualquier cambio futuro quede documentado y pueda aplicarse mediante migraciones controladas. Con esta base, queda establecido el esqueleto técnico necesario para avanzar hacia el desarrollo completo del proyecto, integrando el backend y posteriormente el frontend, garantizando que ambos interactúen de forma adecuada con un modelo de datos bien estructurado y preparado para crecer.